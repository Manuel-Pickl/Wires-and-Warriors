\chapter{Einführung und Idee}

Das Spiel `Wire \& Warriors' ist ein innovative Spiel, dass die Intensität eines Escape Rooms mit der Geschicklichkeit eines `Heißen Draht'-Spieles vereint. Dieses Spiel ist eine neue Herangehensweise an das klassiche Konzept dar, indem es Teamarbeit und Geschicklichkeit miteinander verbindert.

\section{Spielkonzept eines Escape-Room}

Ein Escape-Room basiert auf dem Prinzip der Live-Action-Rätselspiele, bei denen Spieler in einem speziell gestalteten Raum oder einer Umgebung eingeschlossen sind. In diesem Raum müssen verschiedene Rätsel gelöst werden, um zu entkommen. Im folgenden werden die Schlüsselkomponenten aufgelistet:

\begin{compactitem}
 \item \textbf{Thematische Umgebung}: Ein Escape-Room besitzt ein Thema und zusätzlich eine Geschichte. Das Thema sowie die Geschichte bilden den Rahmen für die Rästel und Aufgaben.
 \item \textbf{Rätsel und Aufgaben}: Die Spieler im Escape-Room müssen eine Reihe von Rätseln lösen, um zu entkommen. Die Rästel bzw. Aufgaben können aus Beobachtungsaufgaben, Geschicklichkeit oder auch Teamarbeit bestehen. Häufig sind die Rästel thematisch in die Umgebung eingebettet und tragen zur Gesamtgeschichte bei.
 \item \textbf{Zeitlimit}: Die Spieler haben eine festgelegte Zeit (oftmals eine Stunde), um alle Aufgaben zu lösen und aus dem Raum zu entkommen. Dadurch soll Druck erzeugt werden und die Spannung des Spiels erhöhen.
 \item \textbf{Teamarbeit}: Escape-Rooms sind oft auf Teamarbeit ausgelegt, um gemeinsam zu rästeln und zu entkommen.
 \item \textbf{Interaktive Elemente}: Escape-Rooms nutzen technologische und mechanische Vorrichtungen, um das Spielerlebnis zu bereichern.
\end{compactitem}

\section{Spielkonzept vom Heißen Draht-Spiel}
Das Heiße Draht-Spiel ist ein klassisches Geschicklichkeitsspiel, bei dem die Spieler einen Metallstab entlang eines gewundenen Metalldrahtes führen müssen, ohne diesen zu berühren. Die Hauptelemente dieses Spiels sind:

\begin{compactitem}
 \item \textbf{Grundkonzept}:  Es wird einen Metallstab oder eine Schlaufe verwendet, um ihn entlang des geformten Drahtes zu bewegen. Berührt der Stab den Draht, gibt es in ein Signal (oft ein Geräusch oder Licht), das einen Fehler anzeigt.
 \item \textbf{Geschicklichkeit und Konzentration}: Das Spiel erfordert eine ruhige Hand und Präzision, um den Draht nicht zu berühren. Die Herausforderung besteht darin, den Stab gleichmäßig entlang des Drahtes zu führen.
 \item \textbf{Verschiedene Schwierigkeitsgrade}: Das Design des Drahtes kann variieren, um verschiedene Schwierigkeitsgrade zu bieten oder zusätzlich erhöht werden durch eine kleinere Schlaufen.
 \item \textbf{Wettbewerbs- und Zeitfaktor}: Es ist möglich die Spielzeit zu messen, um zu sehen, wie schnell ein Spieler das Ende erreicht hat. Dadurch ist es möglich den Wettbewerbsaspekt hinzuzufügen.
 \item \textbf{Einfachheit und Zugänglichkeit}: Eines der Hauptmerkmale des Heißen Draht-Spiels ist seine Einfachheit in Bezug auf die Regeln und die leichte Zugänglichkeit, was es zu einem beliebten Spiel für alle Altersgruppen macht.
\end{compactitem}

\section{Spielkonzept von Wire \& Warriors}

Das Spielkonzept von Wire \& Warriors stellt eine Fusion der zwei Spielarten dar: des Escape-Rooms und des Heißen Draht-Spiels. Dadurch konnte ein innovatives Spielprinzip entwickelt werden, dass sowohl die kognitiven als auch die feinmotorischen Fähigkeiten der Spieler auf die Probe stellt. Die Hauptelemente des Spiele sind:

\begin{compactitem}
 \item \textbf{Integration von Escape-Room-Elementen}: Im Spiel Wire \& Warriors müssen die Spieler dem Heißen Draht-Spiel entkommen. Dies ist nur möglich, indem sie durch die verschiedenen Levels fortschreiten. 
 \item \textbf{Einbindung des Heißen Draht-Prinzips}: Im Kern des Spiels steht das Konzept des Heißen Draht-Spiels, bei dem Spieler einen Metallstab entlang eines verwickelten Drahtes führen müssen, ohne diesen zu berühren.
 \item \textbf{Innovative Levels und Herausforderungen}: Das Spiel erhöht den Schwierigkeitsgrad mit fortschreitenden Levels und eine Brücke die gemeinsam überwunden werden muss. Jedes Level bringt neue, komplexere Drahtkonfigurationen, was die Spieler dazu zwingt, ihre Geschicklichkeit zu verbessern.
 \item \textbf{Interaktive und immersive Erfahrung}: Durch Technologische Elemente wie Licht- und Soundeffekte verstärken die Gesamterfahrung.
 \item \textbf{Anpassungsfähigkeit und Vielfalt}: Das Spiel ist so konzipiert, dass es variable Schwierigkeitsgrade gibt und dadurch die Schwierigkeit erhöht werden kann indem z.B. es eine kleinere Schlaufen zum spielen nutzen. Auch ist das gesamte Spiel modular gestaltet, sodass alle Spielelemente jederzeit ausgetauscht werden können, um das Spielerlebnis anzupassen.
\end{compactitem}

\subsection{Spielregeln}

Das Spiel beginnt, wenn beide Spieler ihre Schlaufen in der Startposition haben. Die Startposition ist erreichbar, indem beide Spieler ihre Schlaufe durch den Draht am Turm fädeln. Die Positionen sind gültig, wenn die Platten am Turm berührt werden und das grüne Licht am Turm aufleuchtet. Die erfolgreiche Positionierung wird durch einen Sound signalisiert, daraufhin können die Spieler mit dem Spiel beginnen.

\textbf{Ziel des Spiels}: Das Hauptziel ist es, die gegenüberliegende Seite zu erreichen, ohne dabei den Draht zu berühren. Sobald ein Spieler den gegenüberliegenden Turm erreicht, muss er die Platten am Turm erneut mit seiner Schlaufe berühren. Die erfolgreiche Positionierung wird erneut durch ein grünes Licht am Turm signalisiert.

\subsubsection{Level und Herausforderung}

\begin{compactitem}
 \item \textbf{Level 1}: Im ersten Level bewegt sich nur die zentrale Brücke des Spiels, während die Drähte auf der linken und rechten Seite statisch bleiben.
 \item \textbf{Level 2}: Im zweiten Level beginnen alle Drähte sich zu bewegen, allerdings in einer langsameren Geschwindigkeit.
 \item \textbf{Level 3}: Im dritten Level erhöht sich die Geschwindigkeit der Drähte an den beiden Türmen.
\end{compactitem}

\textbf{Lebenspunkte und Herzsystem}: Das Spiel verfügt über ein Herzsystem, dass insgesamt aus sechs Herzen besteht. Jedes Mal, wenn ein Spieler einen Draht berührt, verliert das Team ein Herz. Die Herzen sind auf dem Spielbrett zusehen und werden durch rote LEDs signalisiert. Verliert man ein Herz, erlischt die entsprechende LED. In jedem neuen Level werden die Herzen zurückgesetzt auf sechs Herzen.

\textbf{Fehler und Konsequenzen}: Wenn die Schlaufe mit einem Draht in Berührung bleibt, zieht das Spiel kontinuierlich Herzen ab. Sind alle Herzen verloren, muss das Spiel neu gestartet werden.

\textbf{Spielabschluss}: Ist das Spiel erfolgreich abgeschlossen worden, erhalten die Spieler am Ende einen geheimen Code als Belohnung.