\chapter{Ausblick und Fazit}

\section{Ausblick}

Das aktuelle Design des Spiels ist im Großen und Ganzen gut durchdacht und funktioniert einwandfrei. Jedoch haben wir eine Schwachstelle identifiziert, und zwar im Bereich des Brückensegments. Dieses Element ist essentiell für das Spiel, da es den Spielern ermöglicht, einander zu passieren und zur gegenüberliegenden Seite zu gelangen. Gleichzeitig stellt es jedoch einen kritischen Punkt dar, an dem die Fairness der Spieler gefordert ist. Theoretisch könnten Spieler diese Stelle nutzen, um aus dem vorgegebenen System auszubrechen und das Spiel auf eine unfaire Weise zu beenden.

Wir haben verschiedene Entwürfe und Konzepte für einen Tunnel erarbeitet, der das Überqueren der Brücke erlaubt, ohne die Möglichkeit zu bieten, das System zu umgehen. Leider konnte keine dieser Ideen das Problem zufriedenstellend lösen. Letztendlich haben wir uns entschieden, die Situation so zu belassen, wie sie ist, und setzen auf das faire Verhalten der Spieler.

Diese Entscheidung spiegelt einen Kompromiss wider, der oft in der Spielentwicklung gemacht werden muss, zwischen idealer Designlösung und praktischer Umsetzbarkeit. Wenn in Zukunft an dem Spiel weitergearbeitet wird, bietet die Lösung dieses Problems eine sinnvolle Zielsetzung.

Eine Überarbeitung des Designs könnte außerdem eine wertvolle Verbesserung für das Spiel darstellen. Die derzeitige Konstruktion aus dünnen Holzplatten, welche zusammengesteckt und teilweise mit Heißkleber und Flachwinkeln verstärkt wurden, bietet zwar die grundlegende Funktionalität, jedoch gibt es sowohl in Bezug auf Stabilität als auch auf Ästhetik Raum für Verbesserungen.

Eine einfache und kosteneffiziente Möglichkeit wäre das Färben oder Lackieren der Holzplatten. Dies würde nicht nur das ästhetische Erscheinungsbild verbessern, sondern könnte auch die Haltbarkeit des Holzes erhöhen. Verschiedene Farben oder Lackierungen könnten auch dazu beitragen, verschiedene Spielbereiche visuell zu unterscheiden. Eine umfassendere Überarbeitung könnte den Ersatz der Holzplatten durch stabilere und langlebigere Materialien beinhalten.

\section{Fazit}

\textbf{Es war ein anspruchsvolles, aber ungemein wertvolles Projekt!}

Die Projekte sind eigentlich für drei Personen ausgelegt, und leider haben wir schon früh im Projektverlauf einen Teamkollegen verloren, sodass wir nur noch zu zweit waren. Als Medieninformatiker war der Bereich der Hardware für uns ein völlig neues Terrain. Wir hatten kaum Erfahrung und verfügten über wenig Vorwissen, insbesondere in grundlegenden Aspekten der Elektrotechnik. Zwar hatten wir im Studium Module wie 'Grundlagen digitaler Systeme' oder 'Mobile and Ubiquitous Computing', doch diese sind schon einige Zeit her, und die relevanten Bereiche für dieses Projekt wurden nur teilweise abgedeckt. Da der Rest des Studiums wenig Berührungspunkte mit diesen Themen aufwies, gingen viele Kenntnisse leider wieder verloren. Daher standen wir vor der Herausforderung, uns das nötige Wissen von Grund auf anzueignen.

Besonders herausfordernd war für uns das Verständnis grundlegender Konzepte der Elektronik, insbesondere von Strom, Spannung und Widerstand. Diese Aspekte waren jedoch essentiell für das Funktionieren unseres Spiels. Glücklicherweise erhielten wir vom Team des MakerSpace wertvolle Unterstützung und Nachhilfe in diesen Bereichen, was uns enorm weiterhalf.

Ein weiteres signifikantes Problem stellte die Inbetriebnahme und Steuerung der Motoren dar, insbesondere im Zusammenhang mit den Schrittmotoren (Stepper-Motoren). Ursprünglich planten wir, kleinere Steuerungsboards einzusetzen, aber wir mussten feststellen, dass unsere Kenntnisse dafür nicht ausreichten. Schließlich entschieden wir uns für den Einsatz größerer Boards, da wir nur mit diesen eine funktionierende Lösung realisieren konnten.

Diese große Aufgabe war jedoch eine Erfahrung, die uns sehr wertvoll erscheint. Auf der Softwareseite traten glücklicherweise keine Probleme auf, was uns sehr zugute kam. Trotz der Herausforderungen und des reduzierten Teams gelang es uns, ein Projekt zu realisieren, das unsere anfänglichen Erwartungen nicht nur erfüllte, sondern in vielerlei Hinsicht sogar übertraf. Diese Erfahrung hat nicht nur unsere Fähigkeiten erweitert, sondern uns auch als Team zusammen geschweißt.